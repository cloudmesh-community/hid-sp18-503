% status: 30
% chapter: TBD

\title{Swager Service for an Openstack Deployment}


\author{Arnav Arnav}
\affiliation{%
  \institution{Indiana University}
  \city{Bloomington} 
  \state{IN} 
  \postcode{47408}
  \country{USA}}
\email{aarnav@iu.edu}


% The default list of authors is too long for headers}
\renewcommand{\shortauthors}{Arnav}


\begin{abstract}
Openstack is one of the most important cloud coputing Infrastructure
as a Service that allows users to setup and manage servers easily. It
provides APIs that help users to manage these services through
command-line and through code. The NIST Big Data Reference
Architecture defines a reference architecture for various objects that
are important in the context of cloud computing. In this project we
aim to build a REST service using these object specifications that
help users complete an openstack deployment easily.
\end{abstract}

\keywords{hid-sp18-503, Swagger, Openstack, API, Cloud}


\maketitle

\section{Introduction}
Representational State Transfer (REST) is an architectural standard
for web applications that allows different applications to comunicate
easily. It allows separation of clients and servers and the server
funcionality can be esily changed without affecting the client's
interface to the server. REST is highly scalable and easy to
extend~\cite{hid-sp18-503-REST}. The NIST Big Data Reference
Architecture defines standard properties for objects that that are
important to Big Data echosystem.%~\cite{}.
These objects can be easily accessed or transferred with the use of
RESTful Applications
that use REST verbs (GET, PUT, POST, UPDATE, DELETE) to manipulate
these objects~\cite{hid-sp18-503-REST}.

Openstack is a free and open source cloud Infrastructure as a Service
that allows users to set up resources such as virtual servers for
their customers. It is one of the most widely deployed cloud
infrastructure and is constantly improving.
%\cite{}

We aim to develop a REST service with the help of Swagger - a tool
that converts a yml specification to code - that can be used by other
applications to setup and manage an openstack deployment.

Doing this requires identification of various objects that are needed
to accomplish such a task. Some of these are images, virtual machines,
virtual directories, files and openstack user details. Each of these
objects should be accessible freom the REST application, so that they
can be updated and stored as necessary. It is also important to have a
REST functionality that allows execution of commands to the openstack
server created with the help of these objects.

We finally create a command line utility using cloudmesh cmd5
that allows users to start up the REST service and run commands
directly from the comandline without worrying too much about the
underlying architecture.

\section{Object Definitions}

\begin{itemize}
\item files : files that may be required by the user or an
application

\item virtual directory : a directroy that has information
about the files. this can be a remote directory that can be accessed
when necessary.

\item image: operating system image that can be used
for a virtual machine

\item virtual machine: the harware details
required for the the virtual machine along with the name and ip
adresses associated.

\item openstack credentials: credentials
provided by openstack to use the openstack api

\item user details :
details about the various users that can access the virtual machine

\end{itemize}

\bibliographystyle{ACM-Reference-Format}
\bibliography{report} 
